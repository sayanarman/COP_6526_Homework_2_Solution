\documentclass[12pt,letterpaper, onecolumn]{exam}
\usepackage{amsfonts, amsmath, amssymb}
\usepackage{amssymb}
\usepackage{commath}
\usepackage{physics}
\usepackage{multirow}
\usepackage{float}
\usepackage{relsize}
\usepackage{tikz}
\usepackage[lmargin=71pt, tmargin=1.2in]{geometry}  %For centering solution box
\usepackage{clrscode}
\usepackage{listings}
\usepackage{xcolor}
\usepackage{pdfpages}
\usepackage{enumitem}
\definecolor{codegreen}{rgb}{0,0.6,0}
\definecolor{codegray}{rgb}{0.5,0.5,0.5}
\definecolor{codepurple}{rgb}{0.58,0,0.82}
\definecolor{backcolour}{rgb}{0.95,0.95,0.92}
\usetikzlibrary{arrows.meta}
\usetikzlibrary{patterns}
\graphicspath{ {./images/} }

\lstdefinestyle{mystyle}{
    backgroundcolor=\color{backcolour},   
    commentstyle=\color{codegreen},
    keywordstyle=\color{magenta},
    numberstyle=\tiny\color{codegray},
    stringstyle=\color{codepurple},
    basicstyle=\ttfamily\footnotesize,
    breakatwhitespace=false,         
    breaklines=true,                 
    captionpos=b,                    
    keepspaces=true,                 
    numbers=left,                    
    numbersep=1pt,                  
    showspaces=false,                
    showstringspaces=false,
    showtabs=false,                  
    tabsize=2
}

%\lstset{style=mystyle}
\lhead{COP 6526 Homework 2 Solution\\}
\rhead{Arman Sayan\\}
% \chead{\hline} % Un-comment to draw line below header
\thispagestyle{empty}   %For removing header/footer from page 1

\begin{document}

\begingroup  
    \centering
    \LARGE COP 6526\\
    \LARGE Homework 2 Solution\\[0.5em]
    \large \today\\[0.5em]
    \large Arman Sayan\par
\endgroup
\rule{\textwidth}{0.4pt}
\bracketedpoints   %Self-explanatory
\printanswers
\renewcommand{\solutiontitle}{\noindent\textbf{Ans:}\enspace}   %Replace "Ans:" with starting keyword in solution box
\qformat{\large \textbf{\thequestion \quad \thequestiontitle \quad [\thepoints] \hfill}}
\renewcommand{\thepartno}{\arabic{partno}}
\renewcommand{\partlabel}{\thepartno.}

\begin{questions}
    \titledquestion{Question 1: Parallel k-Means Clustering}[50]

    Dataset: MNIST digits (subset 10,000 images) (or sklearn.datasets.load\_digits)

    \begin{parts}
        \part[12] Implement k-means clustering (sequential code). k=10.

        \begin{solution}

            Please check the source code included in the .zip file named as

            \begin{center}
                \textbf{cop\_6526\_hw2\_q1\_sequential\_arman\_sayan.py}
            \end{center}
            
            for the sequential implementation of k-means clustering.

        \end{solution}

        \part[12] Parallelize both  E-step (assignment) and M-step (update) using Python Multiprocessing. 
        You may split dataset among processes. During the E-step (assignment), each process assigns points to nearest centers. 
        During the M-step (update),  each process handles a cluster by computing the mean of its assigned points.

        \begin{solution}

            Please check the source code included in the .zip file named as

            \begin{center}
                \textbf{cop\_6526\_hw2\_q1\_multiprocessing\_arman\_sayan.py}
            \end{center}

            for the parallel implementation of k-means clustering using Python Multiprocessing.

        \end{solution}

        \part[14] Parallelize both  E-step (assignment) and M-step (update) using mpi4py.

        \begin{solution}

            Please check the source code included in the .zip file named as

            \begin{center}
                \textbf{cop\_6526\_hw2\_q1\_mpi\_arman\_sayan.py}
            \end{center}

            for the parallel implementation of k-means clustering using mpi4py.

        \end{solution}

        \pagebreak

        \part[12] Answer the following questions:

        \begin{enumerate}[label=(\alph*)]
            \item From your experiments, does increasing the number of processes in multiprocessing make the program run faster?
            \item Does increasing the number of processes in mpi4py make the program run faster?
            \item Does changing the number of processes have any effect on the accuracy of the K-means clustering results?"
        \end{enumerate}

        \begin{solution}

            The runtime comparison of the three different implementations of k-Means clustering is summarized in Table \ref{tab:runtime_accuracy_kmeans}.
            The experiments were conducted on a local machine with 6 physical CPU cores. Since multiprocessing does use logical CPU cores, which is 12
            in this case, we tested up to 12 processes for the multiprocessing implementation.
            The accuracy of the clustering results was evaluated by comparing the predicted labels with the true labels of the dataset.

            \begin{table}[H]
                \centering
                \begin{tabular}{|c|c|c|c|}
                    \hline
                    \textbf{Method} & \textbf{\# Processes} & \textbf{Runtime (s) ↓} & \textbf{Accuracy (\%) ↑} \\
                    \hline
                    \multirow{1}{*}{\texttt{sequential}} 
                        & 1  & \textbf{22.42} & 46.80 \\
                    \hline
                    \multirow{6}{*}{\texttt{multiprocessing}}
                        & 1  & 36.20 & 46.80 \\
                        & 2  & 23.91 & 46.80 \\
                        & 3  & 22.19 & 46.80 \\
                        & 4  & 20.66 & 46.80 \\
                        & 5  & 18.37 & 46.80 \\
                        & 6  & 20.19 & 46.80 \\
                        & 7  & 19.34 & 46.80 \\
                        & 8  & 18.62 & 46.80 \\
                        & 9  & 18.33 & 46.80 \\
                        & 10 & 19.35 & 46.80 \\
                        & 11 & 18.98 & 46.80 \\
                        & 12 & \textbf{18.29} & 46.80 \\
                    \hline
                    \multirow{6}{*}{\texttt{mpi4py}} 
                        & 1 & 35.22 & 47.40 \\
                        & 2 & 13.28 & 47.24 \\
                        & 3 & 7.57  & 49.58 \\
                        & 4 & 6.35  & 48.20 \\
                        & 5 & 9.28  & 42.60 \\
                        & 6 & \textbf{5.93} & 49.78 \\
                    \hline
                \end{tabular}
                \caption{Runtime and Accuracy of k-means clustering using different implementations and number of processes.}
                \label{tab:runtime_accuracy_kmeans}
            \end{table}

            \begin{enumerate}[label=(\alph*)]
                \item The results show that increasing the number of processes in the multiprocessing implementation generally reduces 
                runtime compared to using a single process, although the benefit is not linear. With one process, multiprocessing is slower 
                than the sequential version due to process management overhead. However, as more processes are added, performance improves, 
                and the runtime drops steadily, reaching its best value at 12 processes. After around 5–6 processes, the gains become smaller,
                and runtimes fluctuate slightly, reflecting overhead and diminishing returns. This indicates that multiprocessing can provide 
                speedups, but its efficiency is limited by overhead and the shared-memory model.

                \item The MPI implementation demonstrates far stronger scaling behavior than multiprocessing. 
                Runtime decreases sharply as the number of processes increases, falling from over 35 seconds with one process to under 
                6 seconds with six processes. This shows that MPI is highly effective at distributing the workload across processes. 
                However, the results also reveal a non-linear trend like at five processes, the runtime increases compared to four, 
                likely due to communication overhead or load imbalance. Overall, the results highlight that MPI achieves significant 
                performance gains with more processes, but its efficiency can vary depending on how evenly the work and communication 
                costs are distributed.

                \item Changing the number of processes does not significantly affect the accuracy of k-Means clustering. 
                In the multiprocessing implementation, accuracy remains constant at 46.80\% regardless of process count, showing that parallelism 
                only impacts runtime. In the MPI implementation, accuracy fluctuates slightly between 42–50\%, but these differences are due to 
                the non-deterministic order of reductions, not the number of processes itself. We tried to use the same seed number for all runs 
                to somehow control the random initialization of centroids. Therefore, we conclude that the number of processes 
                influences execution time but has no significant impact on the clustering accuracy of the algorithm.
            \end{enumerate}

        \end{solution}

    \end{parts}

    \pagebreak

    \titledquestion{Question 2}[50]

    Please parallelize the following nonlinear regression program using multiprocessing and MPI in python. 
    According to the number of CPU cores in your computer (or one computer on stokes), you may use the same number of processes.

    Here are the nonlinear regression program and data.

    Hints:

    \begin{enumerate}[label=(\theenumi)]
        \item To parallelize the training, you can distribute the data to different processes at the beginning. 
        Then under every epoch, every process calculates and summarizes gradients, and then sends them to the main process. 
        Then, in the current epoch, the main process collects gradients from all processes, updates the model, and then propagates
         the model to every process. Here we do not consider mini-batch. Such kind of distributed/parallel machine learning is 
         "synchronized", which produces the same result as sequential execution.
        \item Please use Numpy for efficiency.
    \end{enumerate}

    \begin{parts}
        \part[20] Multiprocessing in Python

        \begin{solution}

            Please check the source code included in the .zip file named as

            \begin{center}
                \textbf{cop\_6526\_hw2\_q2\_multiprocessing\_arman\_sayan.py}
            \end{center}

            for the parallel implementation of nonlinear regression program using Python Multiprocessing.

        \end{solution}

        \part[20] MPI in Python

        \begin{solution}

            Please check the source code included in the .zip file named as

            \begin{center}
                \textbf{cop\_6526\_hw2\_q2\_mpi\_arman\_sayan.py}
            \end{center}

            for the parallel implementation of nonlinear regression program using MPI.

        \end{solution}

        \part[10] Please compare the entire runtime of the three different implementations 
        (sequential, multiprocessing, MPI) in a report.

        \begin{solution}

        \end{solution}

    \end{parts}

\end{questions}

\end{document}